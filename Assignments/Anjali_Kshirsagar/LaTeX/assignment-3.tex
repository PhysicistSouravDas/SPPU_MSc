% !TEX TS-program = pdflatex
% !TEX encoding = UTF-8 Unicode

% This is a simple template for a LaTeX document using the "article" class.
% See "book", "report", "letter" for other types of document.

\documentclass[11pt]{article} % use larger type; default would be 10pt

\usepackage[utf8]{inputenc} % set input encoding (not needed with XeLaTeX)
\usepackage{amsmath} % for align
\usepackage{mathtools} % for aboxed
\usepackage{amssymb} % therefore etc

%%% Examples of Article customizations
% These packages are optional, depending whether you want the features they provide.
% See the LaTeX Companion or other references for full information.

%%% PAGE DIMENSIONS
\usepackage{geometry} % to change the page dimensions
\geometry{a4paper} % or letterpaper (US) or a5paper or....
\geometry{
    left = 0.9in,   % Refer: https://www.overleaf.com/learn/latex/Page_size_and_margins
    right = 0.9in,  % for these parameters.
    top = 0.5in,
    bottom = 0.5in
    } % for example, change the margins to 2 inches all round
% \geometry{landscape} % set up the page for landscape
%   read geometry.pdf for detailed page layout information

\usepackage{graphicx} % support the \includegraphics command and options
% \graphicspath{
%     {../Assignment-1/}
% }
% \usepackage{wrapfig}

% \usepackage[parfill]{parskip} % Activate to begin paragraphs with an empty line rather than an indent

%%% PACKAGES
\usepackage{booktabs} % for much better looking tables
\usepackage{array} % for better arrays (eg matrices) in maths
\usepackage{paralist} % very flexible & customisable lists (eg. enumerate/itemize, etc.)
\usepackage{verbatim} % adds environment for commenting out blocks of text & for better verbatim
\usepackage{subfig} % make it possible to include more than one captioned figure/table in a single float
% These packages are all incorporated in the memoir class to one degree or another...

%%% HEADERS & FOOTERS
\usepackage{fancyhdr} % This should be set AFTER setting up the page geometry
\pagestyle{fancy} % options: empty , plain , fancy
\renewcommand{\headrulewidth}{0pt} % customise the layout...
\lhead{}\chead{}\rhead{}
\lfoot{}\cfoot{\thepage}\rfoot{}

%%% SECTION TITLE APPEARANCE
\usepackage{sectsty}
\allsectionsfont{\sffamily\mdseries\upshape} % (See the fntguide.pdf for font help)
% (This matches ConTeXt defaults)

%%% ToC (table of contents) APPEARANCE
\usepackage[nottoc,notlof,notlot]{tocbibind} % Put the bibliography in the ToC
\usepackage[titles,subfigure]{tocloft} % Alter the style of the Table of Contents
\renewcommand{\cftsecfont}{\rmfamily\mdseries\upshape}
\renewcommand{\cftsecpagefont}{\rmfamily\mdseries\upshape} % No bold!

%%% For displaying nice program (python)
% Refer: https://en.wikibooks.org/wiki/LaTeX/Source_Code_Listings
\usepackage{listings}
\usepackage{color}

\definecolor{dkgreen}{rgb}{0,0.6,0}
\definecolor{gray}{rgb}{0.5,0.5,0.5}
\definecolor{mauve}{rgb}{0.58,0,0.82}
\definecolor{backcolor}{rgb}{0.95,0.95,0.92}
\definecolor{shadowcolor}{rgb}{0.30, 0.41, 0.44}

% Refer: https://en.wikibooks.org/wiki/LaTeX/Source_Code_Listings#Settings
% for all possible settings
\lstset{frame=tb,
  language=Fortran,
  aboveskip=3mm,
  belowskip=3mm,
  showstringspaces=false,
  columns=flexible,
  basicstyle={\small\ttfamily},
  numbers=left,    % where to put the line-numbers; possible values are (none, left, right)
  numberstyle=\tiny\color{gray},
  keywordstyle=\color{blue},
  commentstyle=\color{dkgreen},
  stringstyle=\color{mauve},
  backgroundcolor=\color{backcolor},
  frame=shadowbox,
  rulesepcolor=\color{shadowcolor}
%   breaklines=true,
%   breakatwhitespace=true,
%   tabsize=3
}
%%% End custom block for displaying program
\usepackage{url}
\setlength{\parindent}{0pt} % disabling auto indent in whole document
%%% END Article customizations

%%% The "real" document content comes below...

% \pagenumbering{gobble}
\title{Assignment-3}
\author{Sourav Das (\url{21021085@uopca.unipune.ac.in})}
\date{} % Activate to display a given date or no date (if empty),
         % otherwise the current date is printed 

\begin{document}
\maketitle

\textbf{(a) Discuss the validity of the following groups of characters as FORTRAN variables. If you think they are not valid, give reasons.}
\begin{enumerate}[(i)]
    \item \lstinline{STOP}\\
    \textbf{Ans.} This is \textbf{valid} as variable.
    \item \lstinline{abc-1}\\
    \textbf{Ans.} This is \textbf{not} a valid variable, since hyphen/negative sign is not allowed in variables. It is used as subtraction operator.
    \item \lstinline{1A2B3C}\\
    \textbf{Ans.} This is \textbf{not} a valid variable, since variables cannot start with numbers.
    \item \lstinline{do I 3}\\
    \textbf{Ans.} This is also \textbf{not} a valid variable, since a variable name cannot have spaces between the characters.
    \item \lstinline{FUNCTION}\\
    \textbf{Ans.} This is \textbf{valid} as a variable.
\end{enumerate}
\textbf{(b) Discuss the validity of the following statements as FORTRAN statements. If you think they are not valid, give reasons.}
\begin{enumerate}[(i)]
    \item \lstinline{doI = 2.54}\\
    \textbf{Ans.} This is a \textbf{valid} statement. \lstinline{doI} is assigned to the value \lstinline{2.54}.
    \item \lstinline{STOP = END}\\
    \textbf{Ans.} If the type of \lstinline{STOP} and \lstinline{END} are declared, then \lstinline{STOP} and \lstinline{END} will work as variables instead of FORTRAN keywords, and this statement will assign the value of \lstinline{STOP} variable to that of \lstinline{END}, which is a valid statement. \\
    Otherwise, a keyword cannot be assigned to something else. In that case, this statment is invalid.
    \item \lstinline{WRITE(I, J) = I + J}\\
    \textbf{Ans.} If the type of \lstinline{I, J} are declared, and assigned to a value, and \lstinline{WRITE} is defined as a 2D array, then this statement will perfectly work fine. A minimum working example is given below:
    \lstinputlisting{assignment3mwe.txt}
    Here, 2 is assigned to the element which is present in \lstinline{WRITE(1, 1)}. In line 6, the first \lstinline{write(*,*)} is acting as the keyword, and the second as variable, although FORTRAN is case-insensitive. So the given statement is \textbf{valid} in this case. 

    Otherwise, if \lstinline{WRITE} is not declared in the program as a variable, then it will work as the keyword everywhere in the program, and it cannot be assigned to anything else. In that case, this statement will be \textbf{invalid}.
    \newpage
    \item \lstinline{A + B = C + D} \\
    \textbf{Ans.} This is an \textbf{invalid} statement. Here, \lstinline{=} is acting as an assignment operator. On the left of \lstinline{=}, a single variable can be used, and on the right, an other variable, or a value, or an expression. Here, two variables are present to the left of \lstinline{=}, which is not a valid syntax.
    \item \lstinline{if (I = J) exit}\\
    \textbf{Ans.} This statement is \textbf{invalid}. In the parenthesis beside \lstinline{if}, a \textit{conditional} have to be used, resulting \lstinline{.true.} or \lstinline{.false.}. But here, assignment operation is done, which will result in syntax error. To correct this, \lstinline{==} can be used for comparison of equality of \lstinline{I} and \lstinline{J}.
\end{enumerate}
\textbf{(c)  Evaluate the following arithmetic expressions as Fortran arithmetic expressions (use default type for variables) given I = 6, J = 2, K = 3, L = 9, A = 2.4, B = -3.2, C = 2}.
\begin{enumerate}[(i)]
    \item \lstinline{N = I/J + K/(J + 1)}\\
    \textbf{Ans.} 
    \begin{align*}
        N &= \frac{6}{2} + \frac{3}{(2+1)}  \\
        N &= 3 + \frac{3}{3}    \\
        N &= 3 + 1  \\
        \therefore~\Aboxed{N &= 4}
    \end{align*}
    \item \lstinline{N = I**J**K}\\
    \textbf{Ans.} 
    \begin{align*}
        N &= 6^{2^3}\\
        N &= 6^{8}\\
        \therefore~\Aboxed{N &= 1679616}
    \end{align*}
    \item \lstinline{D = C/I  + A/K}\\
    \textbf{Ans.} 
    \begin{align*}
        D &= \frac{2.0}{6} + \frac{2.4}{3}\\
        D &= 0.33\bar{3} + 0.8\\
        \therefore~\Aboxed{D & = 1.13\bar{3}}
    \end{align*}
    \item \lstinline{E = B**A + K/L}\\
    \textbf{Ans.} 
    \begin{align*}
        E &= -3.2^{2.4} + \frac{3}{9}\\
        E &= \text{\lstinline{NaN}} + 0\\
        \therefore~\Aboxed{E &= \text{\lstinline{NaN}}}
    \end{align*}
    Here, $-3.2^{2.4}$ is a complex number. But since, here, \lstinline{E} is \lstinline{implicit} data type, i.e., real, so it can not store complex data type. Hence, Not a Number (\lstinline{NaN}) is shown.
    \item \lstinline{N = I+J/A*B + C/(J+K)}\\
    \textbf{Ans.} 
    \begin{align*}
        N &= 6+ \frac{2}{2.4}(-3.2) + \frac{2.0}{2+3}\\
        N &= 6 - 2.6\bar{6} + 0.4\\
        N &= 3.7\bar{3}
    \end{align*}
    But since the implicit data type of N is integer, so, the values after decimal places will be truncated.
    \[ \therefore~\boxed{N = 3} ~~~~~~~~~~~~~~~~~~~~~~~~~~~~~~\]
\end{enumerate}
\newpage
\textbf{(d) Give the output of the following program segments (use default type for variables).
}
(i)
\begin{lstlisting}
do I = 1,3
    do J = 1,3
        A(I,J) = I/J *(J/I)
    enddo
    write(*,10) (A(I,K), K = 1,3)
enddo
10 format (5 I 3)
\end{lstlisting}
\textbf{Ans.}
\begin{verbatim}
    1  0  0
    0  1  0
    0  0  1
  
\end{verbatim}
(ii)
\begin{lstlisting}
N = 0
do I = 1,3
    do J = 1,I,2
        do K = 1,J
            do L = 1,K
                N = N + 2
            enddo
        enddo
    enddo
enddo
write(*,10) 'The value of N is', N
10   format(I5)
\end{lstlisting}
\textbf{Ans.} Considering \lstinline{A17} is also included in the format descriptor before \lstinline{I5}, the output is:
\begin{verbatim}
    The value of N is   18

\end{verbatim}
\textbf{(e) Write a program to read the following matrix column-wise and write it as a matrix as it is.}\\
A(1,1) = 1, A(1,2) = 2, A(1,3) = 3 \\
A(2,1) = 4, A(2,2) = 5, A(2,3) = 6  \\
A(3,1) = 7, A(3,2) = 8, A(3,3) = 9 \\
\textbf{Ans.} 
\begin{lstlisting}
program MATRIX_COLUMNWISE
    integer :: A(10,10)
    ! A(1,1) = 1; A(1,2) = 2; A(1,3) = 3
    ! A(2,1) = 4; A(2,2) = 5; A(2,3) = 6
    ! A(3,1) = 7; A(3,2) = 8; A(3,3) = 9

    ! Reading the matrix column-wise
    ! row index I will change more than column index J
    write(*,*) "Enter Order of Square Matrix."
    read(*,*) N
    write(*,*) "Enter Matrix Elements column-wise:"
    do J = 1, N
        do I = 1, N
            read(*,*) A(I, J)
        end do
    end do
    write(*,*) "The matrix is:"
    do I = 1, N
        write(*,*) (A(I, J), J=1,N)
    end do
end program MATRIX_COLUMNWISE
! OUTPUT
! Enter Order of Square Matrix.
! 3
!  Enter Matrix Elements column-wise:
! 1
! 4
! 7
! 2
! 5
! 8
! 3
! 6
! 9
!  The matrix is:
!            1           2           3
!            4           5           6
!            7           8           9


\end{lstlisting}

\end{document}
